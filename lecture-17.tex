% !TEX spellcheck = en_US
% !TEX spellcheck = LaTeX
\documentclass[letterpaper,10pt,english]{article}
\usepackage{%
	amsfonts,%
	amsmath,%	
	amssymb,%
	amsthm,%
	babel,%
	bbm,%
	biblatex,%
	caption,%
	centernot,%
	color,%
	enumerate,%
	%enumitem,%
	epsfig,%
	epstopdf,%
	etex,%
	%framed,%
	fullpage,%
	%geometry,%
	graphicx,%
	hyperref,%
	latexsym,%
	mathptmx,%
	mathtools,%
	multicol,%
	pgf,%
	pgfplots,%
	pgfplotstable,%
	pgfpages,%
	proof,%
	psfrag,%
	subfigure,%	
	tikz,%
	times,%
	ulem,%
	url,%
	xcolor%
}

\usepackage[margin=1in,top=0.75in]{geometry}
\usepackage[mathscr]{eucal}
\usepgflibrary{shapes}
\usepgfplotslibrary{fillbetween}
\usetikzlibrary{%
  arrows,%
  backgrounds,%
  chains,%
  decorations.pathmorphing,% /pgf/decoration/random steps | erste Graphik
  decorations.text,% 
  matrix,%
  positioning,% wg. " of "
  fit,%
  patterns,%
  petri,%
  plotmarks,%
  scopes,%
  shadows,%
  shapes.misc,% wg. rounded rectangle
  shapes.arrows,%
  shapes.callouts,%
  shapes%
}

\theoremstyle{plain}
\newtheorem{thm}{Theorem}[section]
\newtheorem{lem}[thm]{Lemma}
\newtheorem{prop}[thm]{Proposition}
\newtheorem{cor}[thm]{Corollary}

\theoremstyle{definition}
\newtheorem{defn}[thm]{Definition}
\newtheorem{conj}[thm]{Conjecture}
\newtheorem{exmp}[thm]{Example}
\newtheorem{assum}[thm]{Assumptions}

\theoremstyle{remark}
\newtheorem{rem}{Remark}
\newtheorem{note}{Note}

%\newcommand{\norm}[1]{\left\lVert#1\right\rVert}
\newcommand{\tr}{\operatorname{tr}}
\newcommand{\Var}{\operatorname{Var}}
\newcommand{\Cov}{\operatorname{Cov}}
\newcommand{\Real}{\mathbb{R}}

\newcommand{\eq}[1]{\begin{align*}#1\end{align*}}
\newcommand{\meq}[2]{\begin{xalignat*}{#1}#2\end{xalignat*}}
\newcommand{\norm}[1]{\left\lVert#1\right\rVert}
\newcommand{\expect}[1]{\mathbb{E}\left[{#1}\right]}
\newcommand{\prob}[1]{\mathbb{P}\left[{#1}\right]}
\newcommand{\given}{\; \big\vert \;} 
\newcommand{\R}{\mathbb{R}}
\newcommand{\N}{\mathbb{N}}
\newcommand{\Z}{\mathbb{Z}}
\newcommand{\E}{\mathbb{E}}
\renewcommand{\H}{\mathscr{H}}
\renewcommand{\P}{\mathbb{P}}
\newcommand{\sF}{\mathscr{F}}
\newcommand{\sE}{\mathscr{E}}

% a few handy macros

\newcommand\matlab{{\sc matlab}}
\newcommand{\goto}{\rightarrow}
\newcommand{\bigo}{{\mathcal O}}
%\newcommand{\half}{\frac{1}{2}}
%\newcommand\implies{\quad\Longrightarrow\quad}
\newcommand\reals{{{\rm l} \kern -.15em {\rm R} }}
\newcommand\complex{{\raisebox{.043ex}{\rule{0.07em}{1.56ex}} \hskip -.35em {\rm C}}}


% macros for matrices/vectors:

% matrix environment for vectors or matrices where elements are centered
\newenvironment{mat}{\left[\begin{array}{ccccccccccccccc}}{\end{array}\right]}
\newcommand\bcm{\begin{mat}}
\newcommand\ecm{\end{mat}}

% matrix environment for vectors or matrices where elements are right justifvied
\newenvironment{rmat}{\left[\begin{array}{rrrrrrrrrrrrr}}{\end{array}\right]}
\newcommand\brm{\begin{rmat}}
\newcommand\erm{\end{rmat}}

% for left brace and a set of choices
%\newenvironment{choices}{\left\{ \begin{array}{ll}}{\end{array}\right.}
\newcommand\when{&\text{if~}}
\newcommand\otherwise{&\text{otherwise}}
% sample usage:
%  \delta_{ij} = \begin{choices} 1 \when i=j, \\ 0 \otherwise \end{choices}


% for labeling and referencing equations:
\newcommand{\eql}{\begin{equation}\label}
\newcommand{\eqn}[1]{(\ref{#1})}
% can then do
%  \eql{eqnlabel}
%  ...
%  \end{equation}
% and refer to it as equation \eqn{eqnlabel}.  


% some useful macros for finite difference methods:
\newcommand\unp{U^{n+1}}
\newcommand\unm{U^{n-1}}

% for chemical reactions:
\newcommand{\react}[1]{\stackrel{K_{#1}}{\rightarrow}}
\newcommand{\reactb}[2]{\stackrel{K_{#1}}{~\stackrel{\rightleftharpoons}
   {\scriptstyle K_{#2}}}~}


\makeatletter
\def\th@plain{%
  \thm@notefont{}% same as heading font
  \itshape % body font
}
\def\th@definition{%
  \thm@notefont{}% same as heading font
  \normalfont % body font
}
\makeatother
\date{}
\title{Lecture-20: Random Processes}
\author{}

\begin{document}
\maketitle

\section{Stochastic Processes} 

%\begin{defn} 
Let $(\Omega, \sF, P)$ be a probability space. 
For an arbitrary index set $T$ and state space $\sX \subseteq \R$, a \textbf{random process} is a measurable map $X : (\Omega, T) \to \sX$. 
For each $t \in T$, we have $X_t \triangleq \{X(t, \omega): \omega \in \Omega\}$ is a random variable defined on the probability space $(\Omega, \sF, P)$, 
and random process $X$ is a collection of random variables $X = (X_t \in \sX: t \in T)$. 
For each $\omega \in \Omega$, we have a sample path $X_{\omega} \triangleq (X_t(\omega): t \in T)$ of the process $X$. 

%each defined on the same probability space $(\Omega, \sF, P)$ is called a \textbf{random process} 
%\end{defn}

\subsection{Classification}
State space $\sX$ can be countable or uncountable, corresponding to discrete or continuous valued process.  
If the index set $T$ is countable, the stochastic process is called \textbf{discrete}-time stochastic process or random sequence. 
When the index set $T$ is uncountable, it is called \textbf{continuous}-time stochastic process. 
The index set $T$ doesn't have to be time, if the index set is space, and then the stochastic process is spatial process. 
When $T = \R^n \times [0, \infty)$, stochastic process $X(t)$ is a spatio-temporal process. 
\begin{shaded*}
\begin{exmp}
We list some examples of each such stochastic process. 
\begin{enumerate}[i\_]
\item Discrete random sequence: brand switching, discrete time queues, number of people at bank each day.
\item Continuous random sequence: stock prices, currency exchange rates, waiting time in queue of $n$th arrival, workload at arrivals in time sharing computer systems.
\item Discrete random process:  counting processes, population sampled at birth-death instants, number of people in queues.
\item Continuous random process: water level in a dam, waiting time till service in a queue, location of a mobile node in a network.
\end{enumerate}
\end{exmp}
\end{shaded*}

\subsection{Specification}
To define a measure on a random process, we can either put a measure on sample paths, or equip the collection of random variables with a joint measure. 
%We adopt the second approach, and 
We are interested in identifying the joint distribution $F: \R^T \to [0,1]$. 
To this end, for any $x \in \R^T$we need to know
\begin{align*}
F(x) = P\left(\displaystyle {\bigcap_{t \in T}\{\omega \in \Omega: X_t(\omega) \leq x_t\}}\right) = P(\bigcap_{t \in T}X_t^{-1}(-\infty, x_t]) = P \circ X^{-1}\bigtimes_{t \in T}(-\infty, x_t].
\end{align*}
However, even for a simple independent process with countably infinite $T$, 
%When the index set $T$ is infinite, 
any function of the above form would be zero if $x_t$ is finite for all $t \in T$. 
Therefore, we only look at the values of $F(x)$ when $x_t \in \R$ for indices $t$ in a finite set $S$ and $x_t = \infty$  for all $t \notin S$. 
That is, for any finite set $S \subseteq T $we focus on the product sets of the form 
\eq{
\bigtimes_{s \in S}(-\infty, x_s]\bigtimes_{s \notin S}\R. 
}
%This leads to finite-dimensional distributions as defined below. 
%\begin{defn}  
We can define a \textbf{finite dimensional distribution} for any finite set $S \subseteq T$ and $x_S = \{x_s \in \R : s \in S\}$, 
\begin{align*}
F_S(x_S) = P\left(\displaystyle {\bigcap_{s \in S}\{\omega \in \Omega: X_s(\omega) \leq x_s\}}\right) = P(\bigcap_{s \in S}X_s^{-1}(-\infty, x_s]).
\end{align*}
%\end{defn}
Set of all finite dimensional distributions of the stochastic process $\{X_t: t \in T\}$ characterizes its distribution completely.
Simpler characterizations of a stochastic process $X(t)$ are in terms of its moments. 
That is, the first moment such as mean, and the second moment such as correlations and covariance functions. 
\begin{xalignat*}{5}
&m_X(t) \triangleq \E X_t, && R_X(t,s) \triangleq \E X_tX_s,&& C_X(t,s) \triangleq \E (X_t - m_X(t))(X_s-m_X(s)).
\end{xalignat*}

\begin{shaded*}
\begin{exmp}
Some examples of simple stochastic processes. 
\begin{enumerate}[i\_]
\item $X_t = A \cos 2\pi t$, where $A$ is random. 
%The finite dimensional distribution is given by 
%\begin{align*}
%F_S(x) = P\left(\left\{A\cos 2\pi s \leq x_s, s \in S\right\}\right). %P\left(\left\{A \leq \min_{s \in S\setminus\{(2k+1)\frac{\pi}{2}, k \in \Z\}}\frac{x_s}{\cos 2\pi_s}\right\}\right).
%\end{align*}
%The moments are given by 
%\meq{3}{
%&m_X(t) = (\E A)\cos 2\pi t, && R_X(t,s) =  (\E A^2) \cos 2\pi t\cos 2\pi s,&& C_X(t,s) =\text{Var}(A) \cos 2\pi t\cos 2\pi s.
%}
\item $X_t = \cos(2\pi t+ \Theta)$, where $\Theta$ is random and uniformly distributed between $(-\pi, \pi]$. 
%The finite dimensional distribution is given by 
%\begin{align*}
%F_S(x) = P\left(\left\{\cos(2\pi s + \Theta) \leq x_s, s \in S\right\}\right). %P\left(\left\{A \leq \min_{s \in S\setminus\{(2k+1)\frac{\pi}{2}, k \in \Z\}}\frac{x_s}{\cos 2\pi_s}\right\}\right).
%\end{align*}
%The moments are given by 
%\meq{3}{
%&m_X = 0, && R_X(t,s) =  \frac{1}{2}\cos2\pi (t-s),&& C_X(t,s) = R_X(t,s).
%}
\item $X_n = U^n$ for $n \in \N$, where $U$ is uniformly distributed in the open interval $(0,1)$.
\item $Z_t = At +B$ where $A$ and $B$ are independent random variables. 
\end{enumerate}
\end{exmp}
\end{shaded*}

\subsection{Independence}
Recall, given the probability space $(\Omega, \sF, P)$, two events $A, B \in \sF$ are \textbf{independent events} if 
\begin{align*}
P(A\cap B) = P(A)P(B).
\end{align*}
Random variables $X,Y$ defined on the above probability space, are \textbf{independent random variables} if for all $x,y \in \R$
\begin{align*}
P\{X(\omega) \leq x, Y(\omega) \leq y\} = P\{X(\omega) \leq x\}P\{Y(\omega) \leq y\}.
\end{align*}
A stochastic process $X$ is said to be \textbf{independent} if for all finite subsets $S \subseteq T$, we have
\eq{
P(\{X_s \leq  x_s, s \in S\}) = \prod_{s \in S}P\{X_s \leq x_s\}. 
}
Two stochastic process $X, Y$ for the common index set $T$ are \textbf{independent random processes} if for all %$m,n \in \N$ and 
finite subsets $I, J \subseteq T$ %such that $|I| =m, |J| = n$, 
\begin{align*}
P\left(\{X_i \leq x_i, i \in I \}\cap\{Y_j \leq y_j, j \in J\}\right) = P\left(\{X_i \leq x_i, i \in I\}\right)P\left(\{Y_j \leq y_j, j \in J\}\right).
\end{align*}

\subsection{Conditional Expectation and Filtration} 
Let $(\Omega, \sF, P)$ be the probability space.  
Let $X$ be a measurable random variable on this probability space denoted as $X \in \sF$, 
if the event $X^{-1}(-\infty, x] = \{\omega \in \Omega: X(\omega) \leq x\} \in \sF$ for each $x \in \R$.  
Let $\sE \subseteq \sF$ be a $\sigma$-algebra, then the \textbf{conditional expectation} of $X$ given $\sE$ is denoted $\E[X | \sE]$ and is a random variable $Y = \E[X | \sE]$ where
\begin{enumerate}[i\_] 
\item $Y \in \sE$,
\item for each event $A \in \sE$, we have $\E [X1_A] = \E[Y1_A]$. 
%\eq{
%\int_A X dP = \int_A Y dP. 
%}
\end{enumerate}
Intuitively, we think of the $\sigma$-algebra $\sE$ as describing the information we have. 
For each $A \in \sE$, we know whether or not $A$ has occurred. 
The conditional expectation $\E[X|\sE]$ is then the ``best guess'' of the value of $X$ given the information $\sE$. 
Let $X,Y$ be two random variables defined on this probability space. 
Then, the conditional expectation of $X$ given $Y$ is defined as 
\eq{
\E[X|Y] = \E[X | \sigma(Y)]. 
}
A random variable $X$ is \textbf{independent} of the $\sigma$-algebra $\sE$, if for all $x \in \R$ and $A \in \sE$, 
\eq{
\E[1_{\{X \leq x\}}1_A]= P\{X \leq x\}\cap A = P\{X \leq x\}P(A) = \E1_{\{X \leq x\}}\E1_A. 
}
\begin{lem}
Let $(\Omega, \sF, P)$ be a probability space with $\sE \subseteq \sF$ a $\sigma$-algebra. 
If $X \in \sE$ is a random variable, then $\E[X| \sE] = X$. 
\end{lem}
\begin{proof} 
First condition is true by hypothesis, and the second condition holds for any $A \in \sE$. 
\end{proof}
\begin{lem} 
Let $(\Omega, \sF, P)$ be a probability space with $\sE \subseteq \sF$ a $\sigma$-algebra. 
If $X \in \sF$ be a random variable independent of $\sE$.  
Then, $\E[X| \sE]  = \E[X]$. 
\end{lem}
\begin{proof}
This follows since $\E X \in \sE$ and the random variables $X$ and $1_{A}$ are independent for any $A \in \sE$, 
which implies 
\eq{
\E[X1_A] = \E X \E 1_A = \E [(\E X) 1_A]. 
}
\end{proof}
One can partition the state space $\R$ into measurable sets $E_1, E_2, \dots$ for the random variable $X$ defined on the given probability space. 
Then $\Omega_i \triangleq X^{-1}(E_i)$ is a partition of the sample space $\Omega$. 
Let $Y$ be a random variable defined as the partition index for the random variable $X$. 
That is, 
\eq{
Y = \sum_{i \in \N}i \cdot 1_{\{X \in E_i\}}. 
} 
Let $\sE \triangleq \sigma(\Omega_1, \Omega_2, \dots)$, then one can check that $Y \in \sE$ or $\sigma(Y) = \sE$. 
Hence, $\E[X|Y] = \E[X|\sigma(Y)] = \E[X|\sE]$. 
Clearly, $\E[X|Y]$ would be a function of $Y$ and since $Y$ takes countably many values, we have $Z = \E[X|Y]$ taking countably many values, with $Z_i = Z1_{\{Y = i\}}$ being a constant on the corresponding partition $\Omega_i$  of the sample space. 
One can compute this conditional expectation using joint distribution directly as 
\eq{
\E[X|Y=i] = \int_{\R}xdF_{X|Y=i}(x) = \frac{1}{P(\Omega_i)}\int_{E_i}xdF(x) = \frac{\E X1_{\Omega_i}}{P(\Omega_i)}
}
\begin{lem}
Suppose $\{\Omega_i: i \in \N\}$ be a countable partition of the sample space $\Omega$, and $\sE = \sigma(\Omega_1, \Omega_2, \dots)$ is the $\sigma$-field generated by this partition. 
Then, 
\eq{
\E[X | \sE] = \frac{\E[X1_{\Omega_i}]}{P(\Omega_i)} \text{ on } \Omega_i.
}
\end{lem}
\begin{proof}
It is easy to see that the RHS is constant on each partition $\Omega_i$ and hence is measurable with respect to $\sE$. 
Further, for each $\Omega_i \in \sE$, we have
\eq{
\int_{\Omega_i}\frac{\E[X 1_{\Omega_i}]}{P(\Omega_i)}dP = \E[X 1_{\Omega_i}] = \int_{\Omega_i}XdP. 
}
\end{proof}
\begin{cor} $P(A|B)P(B) = P(A\cap B)$.
\end{cor}
\begin{proof}
Taking $X = 1_A$ and $\sE = \{\emptyset, \Omega, B, B^c\}$, from the previous Lemma we get 
\eq{
P(A|B) = \E[1_A| 1_B]= \E[1_A|\sE] = \frac{\E[1_A1_B]}{P(B)} = \frac{P(A\cap B)}{P(B)}.
} 
\end{proof}

\begin{thm}[Bayes' Formula] 
For a $\sigma$-algebra $\sE \subseteq \sF$, and for any events $G \in \sE$ and $A \in \sF$, we have 
\eq{
P(G|A) = \frac{\E[1_GP(A| \sE)]}{\E P(A|\sE)}.
}
\end{thm}
\begin{proof}
It is easy to check that numerator is $\E 1_G\E[1_A|\sE] = \E[1_{A \cap G}|\sE]$. 
It suffices to show that $\E\E[1_A|\sE] = \E 1_A$, which follows from definition. 
\end{proof}
\begin{cor} 
For the countable partition $(\Omega_1, \Omega_2, \dots)$, if the $\sigma$-algebra $\sE = \sigma(\Omega_1, \Omega_2, \dots)$, then for any events $G \in \sE$ and $A \in \sF$, we have 
\eq{
P(\Omega_i|A) = \frac{P(A|\Omega_i)P(\Omega_i)}{\sum_{j \in \N}P(A|\Omega_j)P(\Omega_j)}.
}
\end{cor}
\begin{proof}
Result follows from the fact that $P(A|\sE) \in \sE$ and hence is a constant on each partition $\Omega_j$. % and $\sum_{j \in \N}1_{\Omega_j} = 1$.
\end{proof}

\subsubsection{Filtration}
A net of $\sigma$-algebras $\{\sF_i \subseteq \sF: i \in I\}$ is called a \textbf{filtration} when the index set $I$ is totally ordered and the net is increasing, that is for all $i \leqslant j \in I$ implies $\sF_i \subseteq \sF_j$. 
For a random process $X$ with an ordered index set $T$, we can define a filtration indexed by $T$, such that $\sF_t$ is the information about the process till index $t$, and 
\eq{
\sF_t \triangleq \sigma(X_s, s \leqslant t).
}
\end{document}