\documentclass[12pt]{article}
\usepackage{amsfonts}
\usepackage{amssymb}
\usepackage{amscd}
\usepackage{graphics}
%\usepackage[dvips]{graphicx}
\usepackage{epsfig}
\usepackage{subfigure}
\usepackage{amsmath}
\usepackage{array}
\usepackage{eqparbox}
\usepackage[bookmarks=false]{hyperref}
\usepackage{fancyhdr}
\usepackage{mathrsfs}
\usepackage[shortlabels]{enumitem}
\usepackage{fancyhdr}

\addtolength{\oddsidemargin}{-.750in}% Controls page offset - Left
\addtolength{\voffset}{-.125in}      % Controls page offset - Top
\addtolength{\textwidth}{1.0in}      % Controls Text width
\addtolength{\textheight}{1.125in}       % Controls Text height

\renewcommand{\baselinestretch}{1} % Controls line spacing
\renewcommand{\thefootnote}{\fnsymbol{footnote}}
\renewcommand{\headrulewidth}{0.4pt}
\renewcommand{\footrulewidth}{0.4pt}

\lhead{E2-202 Random Processes}
\rhead{Dept. of ECE, IISc}
\title {\sc Independence and conditional independence : Examples $\&$ Exercises}
\date{Aug. 12 2017}
\author{Prepared By: Karthik and Sahasranand}
\begin{document}
\maketitle 
\pagestyle{fancy}

\begin{enumerate}
\item \emph{An example where three events are pairwise independent, but not jointly independent}\\
	
\par This example is to demonstrate that if there exist $3$ events $A$, $B$ and $C$ such that
\begin{align}
P(A\cap B)&=P(A)\cdot P(B),\nonumber\\
P(B\cap C)&=P(B)\cdot P(C),\nonumber\\
P(A\cap C)&=P(A)\cdot P(C),\nonumber
\end{align}
then it \emph{need not necessarily be true} that $P(A\cap B\cap C)=P(A)\cdot P(B)\cdot P(C)$.

Consider an experiment that involves tossing $3$ coins\footnote{A coin has two faces, named $H,T$. A toss of a coin results in one of the two faces.} $C_{0}$, $C_{1}$ and $C_{2}$. Clearly, the set of all possible outcomes is 
\begin{equation}
\Omega = \{HHH,HHT,HTH,HTT,THH,THT,TTH,TTT\}\nonumber.
\end{equation}
Let $\mathcal{F}$ be the set of \emph{all} subsets of $\Omega$ (clearly this is a $\sigma$-algebra). Let
\begin{align}
A~-~\text{be the event that coin }C_{0}~\text{shows }H\nonumber\\
B~-~\text{be the event that coin }C_{1}~\text{shows }H\nonumber\\
C~-~\text{be the event that coin }C_{2}~\text{shows }H\nonumber,
\end{align}
and let us consider the following assignment of probabilities:\\
\[\centering
\begin{tabular}{|c|c|}
	\hline 
	\rule[-1ex]{0pt}{2.5ex} Events & Probabilities \\ 
	\hline 
	\rule[-1ex]{0pt}{2.5ex} $ \{HHH\} $ & $ \frac{1}{4} $ \\ 
	\hline 
	\rule[-1ex]{0pt}{2.5ex}  $ \{HHT\} $ & $0 $  \\ 
	\hline 
	\rule[-1ex]{0pt}{2.5ex}  $ \{HTH\} $ & $ 0 $  \\ 
	\hline 
	\rule[-1ex]{0pt}{2.5ex}  $ \{HTT\} $ & $ \frac{1}{4} $  \\ 
	\hline 
	\rule[-1ex]{0pt}{2.5ex}  $ \{THH\} $ & $ 0 $  \\ 
	\hline 
	\rule[-1ex]{0pt}{2.5ex}  $ \{THT\} $ & $ \frac{1}{4} $  \\ 
	\hline 
	\rule[-1ex]{0pt}{2.5ex}  $ \{TTH\} $ & $ \frac{1}{4} $  \\ 
	\hline 
	\rule[-1ex]{0pt}{2.5ex}  $ \{TTT\} $ & $ 0 $  \\ 
	\hline 
\end{tabular} 
\]
Then, we have
\begin{align}
P(A)&=\frac{1}{2},~P(B)=\frac{1}{2},~P(C)=\frac{1}{2},\nonumber\\
P(A\cap B)&=(\{HHH\})+P(\{HHT\})=\frac{1}{4}=P(A)\cdot P(B),\nonumber\\
P(B\cap C)&=P(\{THH\})+P(\{HHH\})=\frac{1}{4}=P(B)\cdot P(C),\nonumber\\
P(A\cap C)&=P(\{HHH\})+P(\{HTH\})=\frac{1}{4}=P(A)\cdot P(C),~\text{but}\nonumber\\
P(A\cap B\cap C)&=P(\{HHH\})=\frac{1}{4}\neq P(A)\cdot P(B)\cdot P(C).\nonumber
\end{align}

\item \emph{An example where three events are jointly independent, but not pairwise independent}\\

\par This example is to demonstrate that if there exist $3$ events $A$, $B$ and $C$ such that
\begin{align}
P(A\cap B\cap C)&=P(A)\cdot P(B)\cdot P(C),\nonumber
\end{align}
then it \emph{need not necessarily be true} that 
\begin{align}
P(A\cap B)&=P(A)\cdot P(B),\nonumber\\
P(B\cap C)&=P(B)\cdot P(C),\nonumber\\
P(A\cap C)&=P(A)\cdot P(C),\nonumber
\end{align} 
i.e., at least one of these or all of these could be violated.

Consider the same example as above (with $A$, $B$ and $C$ being the same events as before), but now with the following assignment of probabilities:
\[\centering
\begin{tabular}{|c|c|}
\hline 
\rule[-1ex]{0pt}{2.5ex} Events & Probabilities \\ 
\hline 
\rule[-1ex]{0pt}{2.5ex} $ \{HHH\} $ & $ \frac{1}{8} $ \\ 
\hline 
\rule[-1ex]{0pt}{2.5ex}  $ \{HHT\} $ & $ \frac{2}{8} $  \\ 
\hline 
\rule[-1ex]{0pt}{2.5ex}  $ \{HTH\} $ & $ 0 $  \\ 
\hline 
\rule[-1ex]{0pt}{2.5ex}  $ \{HTT\} $ & $ \frac{1}{8} $  \\ 
\hline 
\rule[-1ex]{0pt}{2.5ex}  $ \{THH\} $ & $ 0 $  \\ 
\hline 
\rule[-1ex]{0pt}{2.5ex}  $ \{THT\} $ & $ \frac{1}{8} $  \\ 
\hline 
\rule[-1ex]{0pt}{2.5ex}  $ \{TTH\} $ & $ \frac{3}{8} $  \\ 
\hline 
\rule[-1ex]{0pt}{2.5ex}  $ \{TTT\} $ & $ 0 $  \\ 
\hline 
\end{tabular} 
\]

It is left as exercise to check that 
\begin{equation}
P(A\cap B\cap C)=P(A)\cdot P(B)\cdot P(C),\nonumber
\end{equation}
but 
\begin{align}
P(A\cap B)&\neq P(A)\cdot P(B),\nonumber\\
P(B\cap C)&\neq P(B)\cdot P(C),\nonumber\\
P(A\cap C)&\neq P(A)\cdot P(C)\nonumber.
\end{align}


\item \emph{An example where two events are (unconditionally) independent but not conditionally independent (when conditioned on a third event)}\\

\par This example is to demonstrate that if there exist $3$ events $A$, $B$ and $C$ such that
\begin{equation}
P(A\cap B)=P(A)\cdot P(B),\nonumber
\end{equation}
then it \emph{need not necessarily be true} that
\begin{equation}
P(A\cap B|C)=P(A|C)\cdot P(B|C).\nonumber
\end{equation}

\par Consider two independent throws of a fair die.\footnote{A die has six faces, numbered $1,2,3,4,5,6$. A throw of a die results in one of the six faces. For a fair die, all faces are \emph{equally likely} to result.} Verify that 
\[
\Omega = \{(1,1), (1,2), \ldots, (1,6), (2,1), \ldots, (2,6), \ldots, (6,1), \ldots, (6,6)\}
\]
Let $\mathcal{F}$ be the set of \emph{all} subsets of $\Omega$.\\ With this, we have specified what $\Omega$ and $\mathcal{F}$ are. Notice that we have also specified $P:\mathcal{F}\to [0,1]$ by saying that we are throwing a ``fair'' die.\\

Let $A$ be the event that the first throw results in a $5$. So, 
\[
A = \{(5,1),(5,2),\ldots,(5,6)\}
\]
Let $B$ be the event that the second throw results in a $3$. Write out what $B$ looks like. What is $A \cap B$?\\

Now, let us calculate $P(A \cap B)$. Recall that the two throws are independent of each other.
\begin{align*}
P(A \cap B) &= \frac{1}{36}\\
&= \frac{1}{6}\cdot\frac{1}{6}\\
&= P(A)\cdot P(B).
\end{align*}
Thus, $A$ and $B$ are \textbf{independent}. Now, let $C$ be the event that the sum of the results of the two throws is equal to $8$. i.e., 
\[
C = \{(2,6),(3,5),(4,4),(5,3),(6,2)\}.
\]
Clearly, 
\begin{align*}
A \cap B \cap C &= \{(5,3)\}\\
A \cap C &= \{(5,3)\} \text{ and, }\\
B \cap C &= \{(5,3)\}.
\end{align*}
Now, conditioned on event $C$, are events $A$ and $B$ independent? 
\begin{align*}
P(A \cap B | C) &= \frac{P(A \cap B \cap C)}{P(C)}\\
&= \frac{1/36}{5/36}\\
&= \frac{1}{5} \\
P(A|C) &= \frac{P(A \cap C)}{P(C)}\\
&= \frac{1/36}{5/36}\\
&= \frac{1}{5}
\end{align*}
Similarly, $P(B|C) = \frac{1}{5}$. Thus, 
\[
P(A \cap B | C) \neq P(A|C)\cdot P(B|C).
\]
Therefore, conditioned on the event $C$, events $A$ and $B$ are \textbf{not \emph{conditionally} independent}.

\item \emph{An example where two events are conditionally independent (conditioned on a third event), but not independent by themselves unconditionally}

\par This example is to demonstrate that if there exist $3$ events $A$, $B$ and $C$ such that
\begin{equation}
P(A\cap B|C)=P(A|C)\cdot P(B|C),\nonumber
\end{equation}
then it \emph{need not necessarily be true} that $P(A\cap B)=P(A)\cdot P(B)$.

\par Consider two coins $C_1$ and $C_2$. \\Let the probability of $C_1$ resulting in $H$ be $1/3$ and that in $T$ be $2/3$. \\Let the probability of $C_2$ resulting in $H$ be $1/4$ and that in $T$ be $3/4$. \\Now, let $C_0$ be a fair coin. i.e., the probability of $C_0$ resulting in $H$ is $1/2$ and that in $T$ is $1/2$.\\

\textbf{We go through the following procedure: Toss $C_0$. If it results in $H$, pick $C_1$ and toss it twice. If the toss of $C_0$ results in $T$, pick $C_2$ and toss it twice. Note that there are three coin tosses in all. Assume that all the tosses are independent of each other.}\\

Let $C$ be the event that the first toss (i.e., toss of $C_0$) results in $H$.\\
Let $A$ be the event that the second toss results in $H$.\\
Let $B$ be the event that the third toss results in $H$.

Now,
\begin{align*}
P(A \cap B |C) &= P(\text{second and third tosses result in H, given that coin } C_1 \text{ is picked})\\
&= \frac{1}{3}\cdot\frac{1}{3}.
\end{align*}
Similarly we can see that
\begin{align*}
P(A|C) &= \frac{1}{3} \text{ and,}\\
P(B|C) &= \frac{1}{3}.
\end{align*}
Therefore,
\[
P(A \cap B | C) = P(A|C)\cdot P(B|C).
\]
So, conditioned on the event $C$, events $A$ and $B$ are \textbf{\emph{conditionally} independent}.
Now, 
\begin{align*}
P(A \cap B) &= P(A \cap B|C)\cdot P(C) + P(A \cap B|C^C)\cdot P(C^C)\\
&= \frac{1}{9} \cdot \frac{1}{2} + \frac{1}{16} \cdot \frac{1}{2}\\
&= \frac{25}{288}.\\
P(A) &= P(A|C)\cdot P(C) + P(A|C^C)\cdot P(C^C)\\
&= \frac{1}{3} \cdot \frac{1}{2} + \frac{1}{4} \cdot \frac{1}{2}\\
&= \frac{7}{24}.
\end{align*}
Similarly, $P(B) = \frac{7}{24}$. Therefore, $P(A)\cdot P(B) = \frac{24.5}{288}$.
\[
P(A \cap B ) \neq P(A)\cdot P(B).
\]
Thus, $A$ and $B$ are \textbf{not independent}.
\end{enumerate}

\par\textbf{ Miscellaneous exercise problems:} 

Assume $(\Omega,\mathcal{F},P)$ is a probability space. All events in the questions below are subsets of $\Omega$.
\begin{enumerate}
\item Show that given any two events $A$ and $B$, $P(A \cup B) = P(A) + P(B) - P(A \cap B)$.
\item Suppose event $A$ is independent of event $B$. Show that the event $A^C$ is independent of $B$. Also, show that $A^C$ is independent of $B^C$.
\item Suppose $P(A_n)=1$ for all $n=1,2,\ldots$. Show that $P\left(\bigcap\limits_{n=1}^\infty A_n\right)  = 1$.
\item Suppose $A$ and $B$ be events such that $P(A)=3/4$ and $P(B)=1/3$. Show that
\[
\frac{1}{12} \stackrel{(a)}{\le} P(A \cap B) \stackrel{(b)}{\le} \frac{1}{3}.
\]
When does inequality $(a)$ hold with equality? When does inequality $(b)$ hold with equality? Give two examples, one where the first inequality $(a)$ holds with equality and the other where the second inequality $(b)$ holds with equality. Also produce an example where strict inequalities hold.
\end{enumerate}
\end{document}
