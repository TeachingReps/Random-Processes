\documentclass[10pt]{article}
\usepackage{epsfig}
\usepackage{amsmath}
\usepackage{amssymb}
\usepackage{amsfonts}
\usepackage{graphicx}
\addtolength{\oddsidemargin}{-.750in}% Controls page offset - Left
\addtolength{\voffset}{-.125in}      % Controls page offset - Top
\addtolength{\textwidth}{1.0in}      % Controls Text width
\addtolength{\textheight}{1.125in}       % Controls Text height
\renewcommand{\baselinestretch}{1} % Controls line spacing
\renewcommand{\thefootnote}{\fnsymbol{footnote}}


\title {\sc Convergence of Sequence of Random Variables - Some exercises}
\author{$E2-202$ - Random Processes, Fall $2017$, ECE, IISc.\\Prepared by - Karthik and Sahasranand}
\date{October 3, 2017}

\begin{document}
\maketitle 

\begin{enumerate}
\item Let $(\Omega,\mathcal{F},P)=([0,1],\mathcal{B}([0,1]),\lambda)$, where $\lambda$ denotes the standard Lebesgue measure on $\mathbb{R}$.
\begin{enumerate}
	\item Let $X_{n}=n\cdot 1_{\left[0,\frac{1}{n}\right]}$ (here, $1_{A}$ denotes the indicator function of the set $A$). Sketch the cdf of $X_{n}$, and show that $X_{n}\stackrel{d}{\longrightarrow}0$. 
	\item Show that $X_{n}$'s as defined above are not independent.
\end{enumerate}

\item If $\sum\limits_{n=1}^{\infty}E[|X_{n}|^{p}]<\infty$ for some $p>0$, then show that $X_{n}\stackrel{a.s.}{\longrightarrow}0$.

\item Let $ X_n $'s be random variables such that $ P(X_n = 0) = \frac{1}{n} = 1 - P(X_n = 1) $ for all $ n $, and let $ X $ be such that
$ P(X = 1) = 1 $. Prove that $X_{n}$ converges to $X$ in distribution and in probability (prove both separately. Do not use the fact that convergence in probability implies that in distribution).

\item Let $(\Omega,\mathcal{F},P)=((0,1],\mathcal{B}([0,1]),\lambda)$, where $\lambda$ denotes the standard Lebesgue measure on $\mathbb{R}$. In each of the cases below, identify the limit and the notion(s) of convergence to this limit.
\begin{enumerate}
	\item $X_{n}(\omega)=n^{2}\omega\cdot 1_{\left(0,\frac{1}{n}\right)}(\omega)$
	\item $X_{n}(\omega)=n\omega-\lfloor n\omega\rfloor$, where $\lfloor x \rfloor$ denotes the largest integer less than or equal to $x$
	\item $X_{n}(\omega)=n\cdot \omega^{n}$.
\end{enumerate}
\item (\emph{Convergence in distribution to convergence in probability}) Suppose $X_{n}\stackrel{d}{\longrightarrow}c$, where $c\in \mathbb{R}$ is a constant. Then, show that $X_{n}\stackrel{i.p.}{\longrightarrow}c$.

\item Let $(\Omega,\mathcal{F},P)$ be a probability space, and let $X_{1}, X_{2},\ldots$ be a sequence of real-valued random variables defined on $(\Omega,\mathcal{F})$. Let $A$ be a Borel set such that $P(X_{k}\in A)=p$ for all $k\geq 1$, and let $Y_{1},Y_{2},\ldots$ be another sequence of random variables defined as
\begin{equation*}
Y_{n}:=\frac{1}{n}~\sum\limits_{k=1}^{n}~1_{\{X_{k}\in A\}}.
\end{equation*} 
In other words, the random variable $(n\cdot Y_{n})$ counts the number of times $X_{k}\in A$ for $1\leq k\leq n$.
\begin{enumerate}
	\item Show that $Y_{n}$ converges to $p$ in probability (do not use weak law of large numbers. Show explicitly using the definition of convergence in probability).
	\item Does $Y_{n}$ converge to $p$ in the mean-squared sense? Justify your answer.
\end{enumerate}

\item Let $X_{1},X_{2},\ldots$ be iid Exp($1$) random variables. Define $Y_{n}:=\max\{X_{1},\ldots,X_{n}\}$.
\begin{enumerate}
	\item Compute the cdf of $Y_{n}$.
	\item Let $a,b\in \mathbb{R}$ such that $0<a<1<b$. Show that
	\begin{align*}
	P(Y_{n}\leq a\log(n)) &\longrightarrow 0~\text{as }n\to \infty\\
	P(Y_{n}\leq b\log(n)) &\longrightarrow 1~\text{as }n\to \infty.
	\end{align*}
	\item Deduce that $\dfrac{Y_{n}}{\log(n)}\stackrel{d}{\longrightarrow}1$.
\end{enumerate} 

\item (\emph{Convergence in distribution need not imply convergence in probability}) Let $X$ be a Ber($0.5$) random variable. For each $n\geq 1$, let $Y_{n}=X$. Let $Y=1-X$.
\begin{enumerate}
	\item Show that $Y_{n}\stackrel{d}{\longrightarrow}Y$.
	\item Show that $Y_{n}$ does not converge to $Y$ in probability.
\end{enumerate}
\end{enumerate}
\end{document}
