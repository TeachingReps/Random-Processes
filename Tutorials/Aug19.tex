\documentclass[11pt]{article}
\usepackage{amsfonts}
\usepackage{amssymb}
\usepackage{amscd}
\usepackage{graphics}
%\usepackage[dvips]{graphicx}
\usepackage{epsfig}
\usepackage{subfigure}
\usepackage{amsmath}
\usepackage{array}
\usepackage{eqparbox}
\usepackage[bookmarks=false]{hyperref}
\usepackage{fancyhdr}
\usepackage{mathrsfs}
\usepackage[shortlabels]{enumitem}
\usepackage{fancyhdr}

\addtolength{\oddsidemargin}{-.750in}% Controls page offset - Left
\addtolength{\voffset}{-.125in}      % Controls page offset - Top
\addtolength{\textwidth}{1.0in}      % Controls Text width
\addtolength{\textheight}{1.125in}       % Controls Text height

\renewcommand{\baselinestretch}{1} % Controls line spacing
\renewcommand{\thefootnote}{\fnsymbol{footnote}}
\renewcommand{\headrulewidth}{0.4pt}
\renewcommand{\footrulewidth}{0.4pt}

\lhead{E2-202 Random Processes}
\rhead{Dept. of ECE, IISc}
\title {\sc Random Variables and Distribution Functions : Exercises}
\date{Aug. 19 2017}
\author{Prepared By: Karthik and Sahasranand}
\begin{document}
\maketitle 
\pagestyle{fancy}

Assume $(\Omega,\mathcal{F},P)$ is a probability space.
\begin{enumerate}
\item Let $X:\Omega \rightarrow \mathbb{R}$ be a mapping as defined in each of the cases below. For each case, verify that $X$ is indeed a random variable with respect to $\mathcal{F}$ and construct the CDF of $X$.
\begin{enumerate}
	\item $\Omega=\{H,T\}$, $\mathcal{F}=2^{\Omega}$, $P(\{H\})=p$, $X=\mathbf{1}_{\{H\}}$\footnote[2]{$\mathbf{1}_{A}$ stands for the indicator function of the set $A$. $\mathbf{1}_{A}(\omega) = 1$ if $\omega \in A$, $0$ otherwise.}
	\item $\Omega=\{H,T\}$, $\mathcal{F}=2^{\Omega}$, $P(\{H\})=p$, $X= 5 \cdot \mathbf{1}_{\{H\}}$.
	\item $\Omega=\{H,T\}$, $\mathcal{F}=2^{\Omega}$, $P(\{H\})=p$, $X= (-1)^{\mathbf{1}_{\{H\}}}$.
	\item $\Omega = \{1,2,3,4,5,6\}$, $\mathcal{F}$ contains all singletons, $\phi$ and $\Omega$, \\$P(\{\omega\})=1/6$ and $X(\omega)= \omega/2$ for all $\omega \in \Omega$.
	\item $\Omega = \{1,2,3,4,5,6\}$, $\mathcal{F}=\{\phi,\Omega\}$, $X(\omega)= 8$ for all $\omega \in \Omega$.
	\item $\Omega = \{(1,1),(1,2),(1,3),(2,1),(2,2),(2,3),(3,1),(3,2),(3,3)\}$, $\mathcal{F}=2^{\Omega}$, \\$P(\{\omega\})=1/9$ for all $\omega \in \Omega$, \\$X((a,b)) =  a+b$ for all $a,b \in \{1,2,3\}$.
\end{enumerate}
\item Let $X:\Omega\to \mathbb{R}$ be a random variable defined with respect to $\mathcal{F}$. Which of the following are
valid CDF’s of $X$? For each that is not valid, state at least one reason why. For each
that is valid, find $P\left(\{\omega\in \Omega:X(\omega)>5\right\})$ (this is written in short as $P(X>5)$).
\begin{enumerate}[(a)]
	\item 
    $F_{X}(x)=
	\begin{cases}
	\frac{e^{-x^{2}}}{4}&~x< 0\\
	1-\frac{e^{-x^{2}}}{4}&~x\geq 0.
	\end{cases}$
	\item $F_{X}(x)=
	\begin{cases}
	0&~x<0\\
	0.5+e^{-x}&~0\leq x<3\\
	1&~x\geq 3.
	\end{cases}$
    \item $F_{X}(x)=
    \begin{cases}
    0&~x\leq 0\\
    0.5+\frac{x}{20}&~0< x\leq 10\\
    1&~x> 10.
    \end{cases}$
\end{enumerate}
\item Let $X:\Omega\to \mathbb{R}$ be a random variable defined with respect to $\mathcal{F}$. Suppose that $X$ has the following CDF:
\begin{align}
F_{X}(x)=
\begin{cases}
0~&x<-1\\
1-p~&-1\leq x<0\\
1-p+xp~&0\leq x\leq 2\\
1~&x>2\nonumber,
\end{cases}
\end{align}
where $p\in (0,1)$ is a fixed constant. Sketch this function, and find (i) $P(X=-1)$, (ii) $P(X=0)$, and (iii) $P(X\geq 1)$.
\item If $X:\Omega\to \mathbb{R}$ is a random variable defined with respect to $\mathcal{F}$, and $a\in \mathbb{R}$ is a constant, show that $Y=aX$ is also a random variable with respect to $\mathcal{F}$ (hint: use the definition of a random variable). 
\item Take $\Omega=\{1,2,3,4\}$.
\begin{enumerate}[(a)]
	\item Construct a $\sigma$-algebra $\mathcal{F}$ of subsets of $\Omega$ such that 
	\begin{enumerate}[(i)]
		\item $\{1\}\in \mathcal{F}$ and $\{2\}\in \mathcal{F}$
		\item $\mathcal{F}$ is not the power set of $\Omega$.
	\end{enumerate}
	\item Construct a function $X:\Omega\to \mathbb{R}$ that is a random variable with respect to $\mathcal{F}$ constructed in part (a) above.
\end{enumerate}
\item Take $\Omega=\mathbb{R}$. The Borel $\sigma$-algebra of subsets of $\mathbb{R}$, denoted as $\mathcal{B}(\mathbb{R})$ (or $\mathcal{B}$ in short), is the smallest $\sigma$-algebra containing all the open subsets of $\mathbb{R}$. Let $\mathcal{F}=\mathcal{B}$.
\begin{enumerate}
	\item Show that $(-\infty,x]\in \mathcal{F}$ for all $x\in \mathbb{R}$.
	\item Show that $[x,\infty)\in \mathcal{F}$ for all $x\in \mathbb{R}$.
	\item Show that the sets $(1,2),[1,2],(1,2]$ and $[1,2)$ are present in $\mathcal{F}$\\ (more generally, for all $a,b\in \mathbb{R}$ such that $-\infty<a<b<\infty$, $(a,b),[a,b],(a,b]$ and $[a,b)$ are present in $\mathcal{F}$).
	\item Show that $\{x\}\in \mathcal{F}$ for all $x\in \mathbb{R}$.
	\item Show that $\mathcal{F}$ contains all the closed subsets of $\mathbb{R}$.
\end{enumerate}
\end{enumerate}
\vspace{0.5cm}
Reference: The concept of Borel $\sigma$-algebra is taught in a course on measure theory. Prof. Krishna Jagannathan's NPTEL lectures on probability theory, available at \url{http://nptel.ac.in/courses/108106083/1}, explain these concepts in great detail. Interested students may go through these lectures at their convenience. Specifically, lectures $8,9$ and $10$ deal with Borel $\sigma$-algebra and Borel sets.
\end{document}
